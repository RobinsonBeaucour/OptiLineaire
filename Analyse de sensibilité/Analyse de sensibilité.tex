\documentclass{article}
\usepackage[utf8]{inputenc}
\usepackage{amsmath}
\usepackage{amsfonts}
\usepackage{geometry}
\usepackage{stmaryrd}
\usepackage[table]{xcolor}
\usepackage{fancybox}


\geometry{a4paper,left=25mm,right=25mm,top=20mm}
\title{Analyse de sensibilité - Optimisation linéaire}
\author{Robinson Beaucour}
\date{Octobre 2022}

\begin{document}

\maketitle

\section{Problème à résoudre}
On cherche à minimiser le coût de production d'électricité sur une journée.
Les données du problèmes sont ci-dessous.
\rowcolors{2}{lightgray}{gray}
\begin{center}
\begin{tabular}{|m{3cm}|m{5.5cm}|m{3cm}|}
    \hline
    \bf Nom & \bf Description & \bf Type de variable \\
    \hline 
    $t$ & heure de la journée & indice \\
    $X\in\{A,B,C\}$ & type de centrale & indice \\
    $n^{(X)}$ & nombre de centrales de type $X$ & constante \\
    $C^{(X)}_{MWh}$ & Coût de production d'un MWh par une centrale de type $X$ & constante\\
    $P^{(X)}_{max}$ & La puissance maximale d'une centrale de type $X$ & constante\\
    $P^{(X)}_{t}$ & La puissance totale des centrales de type $X$ à l'heure $t$ & variable de décision\\
    \hline
\end{tabular}\\[0.1cm]
\centering \it Récapitulatif des variables du problème\\[0.1cm]
\begin{tabular}{|c|c|c|c|}
    \hline
    \bf Type & \bf $N$ & \bf $P_{max}$ (MW) & \bf $C_{MWh}$ \\
    \hline 
    A & 12 & 2000 & 1.50 \\
    B & 10 & 1750 & 1.38 \\
    C & 5 & 4000 & 2.75 \\
    \hline
\end{tabular}\\[0.1cm]
\centering \it Données des centrales du parc\\[0.1cm]
\begin{tabular}{|c|c|c|c|c|c|}
    \hline
    \bf Heure &0-5&6-8&9-14&15-17&18-23\\
    \hline 
    \bf Consommation(GW) &15&30&25&40&27\\
    \hline
\end{tabular}\\[0.1cm]
\it Données de consommation\\[0.1cm]
\end{center}
On a le problème $\mathcal{P}$ d'optimisation linéaire suivant:
$$
\underset{\forall t, \forall X, P_t^{(X)}\geq 0}{minimize}~ \sum_{t=0}^{23}\sum_{X\in\{A,B,C\}} P_t^{(X)}.C_{MWh}^{(X)}
$$
\rowcolors{0}{}{}
$$
subject~to~ 
\left\{\begin{array}{l}
\forall t \in \llbracket 0,23 \rrbracket,\forall X \in \{A,B,C\},  \\ 
\forall t \in \llbracket 0,23 \rrbracket,  
\end{array}\right.
\left.\begin{array}{l}
P_t^{(X)} \leq n^{(X)} P_{max}^{(X)} \\ 
\underset{X \in \{A,B,C\}}{\sum} P_t^{(X)}\geq d_t
\end{array}\right.
$$
On note $\mathcal{P}_{standard}$ un forme standard du problème $\mathcal{P}$ définit par:\\
$$
\underset{P_t^{(X)}\geq 0, S_t^{(X)}\geq 0,S_t^{(d)}\geq 0}{minimize}~~ \sum_{t=0}^{23}\sum_{X\in\{A,B,C\}} P_t^{(X)}.C_{MWh}^{(X)}
$$
\rowcolors{0}{}{}
$$
subject~to~ 
\left\{\begin{array}{l}
\forall t \in \llbracket 0,23 \rrbracket,\forall X \in \{A,B,C\},  \\ 
\forall t \in \llbracket 0,23 \rrbracket,  
\end{array}\right.
\left.\begin{array}{l}
P_t^{(X)} + S_t^{(X)} = n^{(X)} P_{max}^{(X)} \\ 
\underset{X \in \{A,B,C\}}{\sum} P_t^{(X)} - S_t^{(d)} =  d_t
\end{array}\right.
$$
On pose $X\in \mathbb{R}^{24\cdot7}$, la matrice de blocs de taille $7\times1$.\\
On pose $C\in \mathbb{R}^{24\cdot7}$, la matrice de blocs de taille $7\times1$.\\
On pose $B\in \mathbb{R}^{24\cdot4}$, la matrice de blocs de taille $4\times1$.\\
Dont les blocs sont définis par :
$$
X_t = \left(
\begin{array}{ccccccc}
    P^{(A)}_t\\P^{(B)}_t\\P^{(C)}_t\\S^{(A)}_t\\S^{(B)}_t\\S^{(C)}_t\\S^{(d)}_t
\end{array}
\right),
C_t =\left(
\begin{array}{ccccccc}
    C_{MWh}^{(A)}\\C_{MWh}^{(B)}\\C_{MWh}^{(C)}\\0\\0\\0\\0
\end{array}
\right),
B_t =\left(
\begin{array}{ccccccc}
    P^{(A)}_{max}n^{(A)}\\P^{(B)}_{max}n^{(B)}\\P^{(C)}_{max}n^{(C)}\\d_t
\end{array}
\right),
$$
Enfin on pose $A \in \mathbb{R}^{24\cdot4\times24\cdot7}$, la matrice bloc de taille $4\times7$ :\\[0.2cm]
$$
A_{t,t'} = 0 \text{ si $t\neq t'$}
$$
$$
A_{t,t'} = \left(
\begin{array}{ccccccc}
    1 & 0 & 0 & 1 & 0 & 0 & 0 \\
    0 & 1 & 0 & 0 & 1 & 0 & 0 \\
    0 & 0 & 1 & 0 & 0 & 1 & 0 \\
    1 & 1 & 1 & 0 & 0 & 0 & -1 
\end{array}
\right) \text{ si $t=t'$}
$$
La matrice $A$ est diagonale par bloc.
$\mathcal{P}_{standard}$ peut maintenant s'écrire : 
$$
\underset{X\geq 0}{minimize}~ X\cdot C^{\top}
$$
$$
subject~to~ A \cdot X = B
$$
\begin{center}
    nb\section{Base optimale}
    \boxput*(0,1){
        \colorbox{white}{Rappel de cours}
    }{
    \setlength{\fboxsep}{15pt}
    \fbox{\begin{minipage}{12cm} 
    Soit $\mathcal{P}$ et sa forme standard $\mathcal{P}_{standard}:~min\{cx | Ax = B, x\geq 0\}$.\\
    $\beta$ est une base optimale si et seulement si :\\
    $$
    \left.
    \begin{array}{ccc}
        x_{\beta} &= &A_{\beta}^{-1}b \geq 0, x_{\neg\beta}=0 \\[0.1cm]
        u &= &  c_{\beta}^{\top}A_{\beta}^{-1} \\[0.1cm]
        \overline{c}^{\top} &= &  c^{\top} - u^{\top} A \geq 0
    \end{array}
    \right.
    $$
    \end{minipage}}
    }
\end{center}

On note $\beta_t = (t+1,t+3,t+4,t+5)$ si $t\in \llbracket 0,5 \rrbracket$, $\beta_t = (t,t+1,t+3,t+5)$ sinon. Nous allons montrer que $\beta=\underset{t\in\llbracket 0,23 \rrbracket}{\bigcup}\beta_t$ est une base optimale de $\mathcal{P}_{standard}$.\\
On peut deviner la base grâce aux équivalences suivantes : 
$$
\left.
    \begin{array}{ccc}
        t \text{ appartient à la base}& \Leftrightarrow P_t^{(A)} \text{ est non nul*} &\equiv \text{ Les centrales $A$ produisent à $t$}\\
        t+1 \text{ appartient à la base}& \Leftrightarrow P_t^{(B)} \text{ est non nul*} &\equiv \text{ Les centrales $B$ produisent à $t$}\\
        t+2 \text{ appartient à la base}& \Leftrightarrow P_t^{(C)} \text{ est non nul*} &\equiv \text{ Les centrales $C$ produisent à $t$}\\
        t+3 \text{ appartient à la base}& \Leftrightarrow S_t^{(A)} \text{ est non nul*} &\equiv \text{ Les centrales $A$ ne produisent pas au max possible à $t$}\\
        t+4 \text{ appartient à la base}& \Leftrightarrow S_t^{(B)} \text{ est non nul*} &\equiv \text{ Les centrales $B$ ne produisent pas au max possible à $t$}\\
        t+5 \text{ appartient à la base}& \Leftrightarrow S_t^{(C)} \text{ est non nul*} &\equiv \text{ Les centrales $C$ ne produisent pas au max possible à $t$}\\
        t+6 \text{ appartient à la base}& \Leftrightarrow S_t^{(d)} \text{ est non nul*} &\equiv \text{ Les centrales produisent plus que la demande à $t$}
    \end{array}
\right.
$$
* Sauf si la solution est dégénérée.\\[0.2cm]
On peut raisonner séparément pour chaque $t$ car la matrice $A$ est diagonale par bloc.\\
Pout $t \in \llbracket 0,5 \rrbracket$,\\
$$
A_{t|\beta} = 
\left(
\begin{array}{cccc}
    0 & 1 & 0 & 0 \\
    1 & 0 & 1 & 0 \\
    0 & 0 & 0 & 1 \\
    1 & 0 & 0 & 0 
\end{array}
\right)
,
A_{t|\beta}^{-1} = 
\left(
\begin{array}{cccc}
    0 & 0 & 0 & 1 \\
    1 & 0 & 0 & 0 \\
    0 & 1 & 0 & -1 \\
    0 & 0 & 1 & 0 
\end{array}
\right)
$$
On calcule :
$$
X_{t|\beta} = \left(
\begin{array}{c}
    d_t \\
    n^{(A)}P_{max}^{(A)} \\
    n^{(B)}P_{max}^{(B)} - d_t \\
    n^{(C)}P_{max}^{(C)} \\
\end{array}
\right)\geq 0
,
U_t = \left(
\begin{array}{c}
    0 \\
    0 \\
    0 \\
    C_{MWh}^{(B)} \\
\end{array}
\right),
\overline{C}_t=\left(
\begin{array}{c}
    0 \\
    C_{MWh}^{(A)} - C_{MWh}^{(B)} \\
    C_{MWh}^{(C)} - C_{MWh}^{(B)} \\
    0 \\
    0 \\
    0 \\
    C_{MWh}^{(B)}
\end{array}
\right)\geq 0
$$
Pout $t \in \llbracket 6,23 \rrbracket$,\\
$$
A_{t|\beta} = 
\left(
\begin{array}{cccc}
    1 & 0 & 1 & 0 \\
    0 & 1 & 0 & 0 \\
    0 & 0 & 0 & 1 \\
    1 & 1 & 0 & 0 
\end{array}
\right)
,
A_{t|\beta}^{-1} = 
\left(
\begin{array}{cccc}
    0 & -1 & 0 & 1 \\
    0 & 1 & 0 & 0 \\
    1 & 1 & 0 & -1 \\
    0 & 0 & 1 & 0 
\end{array}
\right)
$$

On calcule :
$$
X_{t|\beta} = \left(
\begin{array}{c}
    d_t - n^{(B)}P_{max}^{(B)} \\
    n^{(B)}P_{max}^{(B)} \\
    n^{(A)}P_{max}^{(A)}+n^{(B)}P_{max}^{(B)} - d_t \\
    n^{(C)}P_{max}^{(C)} \\
\end{array}
\right)\geq 0
,
U_t = \left(
\begin{array}{c}
    0 \\
    C_{MWh}^{(B)}-C_{MWh}^{(A)} \\
    0 \\
    C_{MWh}^{(A)} \\
\end{array}
\right),
\overline{C}_t=\left(
\begin{array}{c}
    0 \\
    0 \\
    C_{MWh}^{(C)} - C_{MWh}^{(A)} \\
    0 \\
    0 \\
    C_{MWh}^{(A)} - C_{MWh}^{(B)} \\
    C_{MWh}^{(A)}
\end{array}
\right)\geq 0
$$

\begin{center}
    \framebox[1.1\width]{On a vérifié que $X_{\beta}\geq 0,\overline{C}\geq 0$. $\beta$ est une base optimale.}
\end{center}
\section{Analyse de sensibilité}
\subsection{Dégénérescence}
Pour tout $t \in\llbracket 0,23 \rrbracket$, $X_{t|\beta} > 0$. La base est donc non-dégénérée.
Pour que la base soit dégénérée il suffit que une des coordonnées de $X_{\beta}$ soit nulle :\\
Pour $t\in\llbracket 0,5 \rrbracket$
$$
\left.
\begin{array}{ccc}
    d_t &= 0&ou\\
    n^{(A)}P_{max}^{(A)} &= 0&ou \\
    n^{(B)}P_{max}^{(B)} &= d_t&ou \\
    n^{(C)}P_{max}^{(C)} & = 0&\\
\end{array}
\right.
$$
Pour $t\in\llbracket 6,23 \rrbracket$
$$
\left.
\begin{array}{ccc}
    d_t & = n^{(B)}P_{max}^{(B)} &ou\\
    n^{(B)}P_{max}^{(B)} & = 0&ou \\
    n^{(B)}P_{max}^{(B)}+n^{(A)}P_{max}^{(A)} &= d_t&ou \\
    n^{(C)}P_{max}^{(C)} & = 0&\\
\end{array}
\right.
$$
On peut donc par exemple mettre $d_t$ ou $n^{(C)}$ à 0. Mais aussi mettre $d_t=n^{(B)}P_{max}^{(B)}$.
\subsection{Augmentation consommation}
On suppose que la consommation augmente de $\delta_t\geq0$. $\beta$ reste optimale si $A^{-1}_{\beta}(b+\delta)\geq 0$, c'est à dire:\\
$$
\forall t \in \llbracket 0,5 \rrbracket, d_t+\delta_t \leq n^{(B)}P_{max}^{(B)}
$$
$$
\forall t \in \llbracket 5,23 \rrbracket, d_t+\delta_t \leq n^{(B)}P_{max}^{(B)} +n^{(A)}P_{max}^{(A)}
$$
Si ces conditions sont respectées, le surcoût peut s'évaluer de cette manière:
$$
\text{Surcoût : }u.\delta = \sum_{t\in\llbracket 0,23 \rrbracket} U_t. (0, 0, 0, \delta_t) = \sum_{t\in\llbracket 0,5 \rrbracket} C_{MWh}^{(B)} \delta_t + \sum_{t\in\llbracket 6,23 \rrbracket} C_{MWh}^{(A)} \delta_t
$$
Ce qui peut être résumé dans ce tableau:
\rowcolors{2}{lightgray}{gray}
\begin{center}
    \begin{tabular}{|c|c|c|}
        \hline
        \bf Période & \bf Surcoût 1 MW par heure & \bf Surcoût max par heure \\
        \hline 
        0-5 & 1.38 & 3450 \\
        6-8 & 1.5 & 17250 \\
        9-14 & 1.5 & 24750 \\
        15-17 &1.5 & 2250 \\
        18-23 & 1.5 & 21750\\
    \hline
    \end{tabular}
\end{center}
\subsection{Augmentation centrale B}
Rien ne change sur la \textbf{période 1} si on ajoute une centrale $B$.$P_t^{(B)}$ reste égal à $d_t$\\
Rappel : 
\rowcolors{0}{}{}
$$
X_{t|\beta} = \left(
\begin{array}{c}
    d_t \\
    n^{(A)}P_{max}^{(A)} \\
    n^{(B)}P_{max}^{(B)} - d_t \\
    n^{(C)}P_{max}^{(C)} \\
\end{array}
\right)=
\left(
\begin{array}{c}
    15 \\
    17.5 \\
    24-15 \\
    20 \\
\end{array}
\right)
\geq 0
$$
Sur la \textbf{période 2}, le coût diminue par une augmentation de $P_t^{(B)}$ et une diminution de $P_t^{(A)}$.
Après avoir vérifié que : \\
$$
X_{t|\beta} = 
\left(
\begin{array}{c}
    d_t - (n^{(B)}+1)P_{max}^{(B)} \\
    n^{(B)}P_{max}^{(B)} \\
    n^{(A)}P_{max}^{(A)}+(n^{(B)}+1)P_{max}^{(B)} - d_t \\
    n^{(C)}P_{max}^{(C)} \\
\end{array}\right)=
\left(
\begin{array}{c}
    30 - 24 \\
    24 \\
    24+17.5 - 30 \\
    20 \\
\end{array}\right)
\geq 0
$$
On constate que $P_t^{(A)}$ passe de $d_t - n^{(B)}P_{max}^{(B)}$ à $d_t - (n^{(B)}+1)P_{max}^{(B)}$. et que $P_t^{(B)}$ passe de $n^{(B)}P_{max}^{(B)}$ à $(n^{(B)}+1)P_{max}^{(B)}$.\\
Le coût varie de $P_{max}^{(B)}\cdot (C_{MWh}^{(B)}-C_{MWh}^{(A)})$ soit 210 d'économies par heure.

\subsection{Diminution Centrale B}
Rien ne change sur la \textbf{période 1} si on retire une centrale $B$.$P_t^{(B)}$ reste égal à $d_t$ car :\\
\rowcolors{0}{}{}
$$
X_{t|\beta} = \left(
\begin{array}{c}
    d_t \\
    n^{(A)}P_{max}^{(A)} \\
    (n^{(B)}-1)P_{max}^{(B)} - d_t \\
    n^{(C)}P_{max}^{(C)} \\
\end{array}
\right)=
\left(
\begin{array}{c}
    15 \\
    17.5 \\
    24-2-15 \\
    20 \\
\end{array}
\right)
\geq 0
$$
Sur la \textbf{période 2}, le coût diminue par une diminution de $P_t^{(B)}$ et une augmentation de $P_t^{(A)}$.
Après avoir vérifié que : \\
$$
X_{t|\beta} = \left(
\begin{array}{c}
    d_t - (n^{(B)}-1)P_{max}^{(B)} \\
    n^{(B)}P_{max}^{(B)} \\
    n^{(A)}P_{max}^{(A)}+(n^{(B)}-1)P_{max}^{(B)} - d_t \\
    n^{(C)}P_{max}^{(C)} \\
\end{array}
\right)=
\left(
\begin{array}{c}
    30 - 24 -2 \\
    24 \\
   17.5+24-30 \\
    20 \\
\end{array}
\right)
\geq 0
$$
En revanche, sur la \textbf{période 3}, la perte de la centrale B rend la base $\beta$ non optimale car on a:
$$
n^{(A)}P_{max}^{(A)}+(n^{(B)}-1)P_{max}^{(B)} - d_t=39.5-40<0
$$
\subsection{Diminution coût centrale C}
Si le coût du MWh des centrales C diminue de 1, la base reste optimale si le coût réduit $\overline{C}$ reste positif. Vérifions :
$$
\forall t \in \llbracket 0,5 \rrbracket,\overline{C}_t=
\left(
\begin{array}{c}
    0 \\
    C_{MWh}^{(A)} - C_{MWh}^{(B)} \\
    C_{MWh}^{(C)}-1 - C_{MWh}^{(B)} \\
    0 \\
    0 \\
    0 \\
    C_{MWh}^{(B)}
\end{array}
\right)=
\left(
\begin{array}{c}
    0 \\
    1.50-1.38 \\
    2.75-1.38 \\
    0 \\
    0 \\
    0 \\
    1.38
\end{array}
\right)
\geq 0
$$
$$
\forall t \in \llbracket 6,23 \rrbracket,\overline{C}_t=
\left(
\begin{array}{c}
    0 \\
    0 \\
    C_{MWh}^{(C)}-1 - C_{MWh}^{(A)} \\
    0 \\
    0 \\
    C_{MWh}^{(A)} - C_{MWh}^{(B)} \\
    C_{MWh}^{(A)}
\end{array}
\right)=
\left(
\begin{array}{c}
    0 \\
    0 \\
    2.75-1-1.50 \\
    0 \\
    0 \\
    1.50-1.38 \\
    1.50
\end{array}
\right)
\geq 0
$$
D'après la description de $\overline{C}$, Si $C_{MWh}^{(B)}$ diminue alors $\overline{C}$ reste positif. Ce qui prouve que la base reste optimale.\\
Le coût du MWh par $B$ peut augmenter jusq'à 1.50 ($=C_{MWh}^{(A)}$) la base $\beta$ restera optimale car on conservera $\overline{C}\geq 0$.
\end{document} 
